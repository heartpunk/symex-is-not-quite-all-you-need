% IEEE Conference Paper - SIF Paper Draft
\documentclass[conference]{./template/IEEEtran}

\usepackage{cite}
\usepackage{amsmath,amssymb,amsfonts}
\usepackage{algorithmic}
\usepackage{graphicx}
\usepackage{textcomp}
\usepackage{xcolor}

\begin{document}

\title{Symbolic Execution is (Not Quite) All You Need\\{\normalsize Version 0.0.1}}

\author{
\IEEEauthorblockN{Sophie Smithburg}
}

\maketitle

\begin{abstract}
TODO
\end{abstract}

\begin{IEEEkeywords}
keyword1, keyword2, keyword3
\end{IEEEkeywords}

% TODO: Add ℐ to notation section (implementation, currently only mentioned inline)
\section{Notation (OUT OF ORDER, TODO: FIND BEST PLACE FOR SECTION)} % chktex 13
\begin{itemize}
\item $\Sigma$ is the concrete runtime space, and in our case, it is parametric over the definition of an ISA\@. It's basically the main memory, registers, CPU flags, any state that is documented for a particular CPU or host machine. In the case of the more abstract view, like a labeled transition system, which we focus on in this paper.
\item $X$ is computed indirectly, but can be defined directly. $X$ is the set of all the state of the subsystem (like the programming language under analysis) that can be controlled by a program in that programming language. It is possible to extend this in the future to include input by way of any predefined input channel, but modeling codata introduces non-determinism, which is out of scope for the current paper.
\item $\pi$ is the mapping from $\Sigma$ to $X$, which we get precisely by the indirect computation mechanism hinted at above.
\item $H$ is the host machine, it can be seen as an LTS or a bog standard ISA
\item $G$ is the true transition system, or notionally, what would be the specification of the programming language if the programming language were defined by the implementation? In some very real sense, a lot of programming languages are. Take, for example, C, Python, and the fact that it's been stated about the specifications in the Python enhancement proposal process and other documents in the standard that if one were to follow them, it's likely, without other guidance, they would end up with an entirely different language.
\item $\tau \in \mathcal{T}$ is a trace, or a sequence of states and labels for the transitions between them, generated when we run the host machine $H$ over the programming language implementation $\mathcal{I}$ and a program from $\mathcal{C}$, which implies all the values relevant to updates (it is of course more convenient when the values for updates are provided for us; we presume this, we think without loss of generality, but perhaps that's wrong!)
\item $\mathcal{T}$ is the set of all traces $\tau$
\item $G'$ is the transition system we extract from traces $\tau \in \mathcal{T}$ of running $H_\mathcal{I}$, i.e.\ the host machine $H$ or host labeled transition system without loss of generality. Additionally, $G'$ is composed from $R$.
\item $\mathcal{I}$ the programming language implementation under analysis
\item $\Gamma$ is the formal grammar for $\mathcal{I}$
\item $\text{holes}(\gamma)$ is the set of holes in production $\gamma$
\item $\kappa_h$ is the sentinel value for hole $h$ in production $\gamma$
\item $\mathcal{C}$ is a covering set of programs that enumerate all productions in $\Gamma$, and alternatives in each production, in a minimized way to be elaborated upon later
\item $L$ is the set of all labels in our transition system; these are derived from the corresponding production under analysis $\gamma \in \Gamma$
\item $\ell \in L$ is a label for a step in our transition system
\item $R_\ell(x, x') := \text{Guard}_\ell(x) \land \text{Update}_\ell(x, x')$ summarizes the preconditions and state transformations for a step labeled $\ell$
\item $R^*$ is the transitive closure of $\bigcup_\ell R_\ell$
\end{itemize}

\section{Introduction}

% TODO: Write introduction using High-Level Goal items from TODO.md

\section*{Acknowledgments}

% TODO: Write acknowledgments

% TODO: Replace \nocite{*} with actual \cite{} commands in text
\nocite{*}
\bibliographystyle{IEEEtran}
\bibliography{refs}

\end{document}
